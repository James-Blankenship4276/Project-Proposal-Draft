
%% bare_jrnl_compsoc.tex
%% V1.4a
%% 2014/09/17
%% by Michael Shell
%% See:
%% http://www.michaelshell.org/
%% for current contact information.
%%
%% This is a skeleton file demonstrating the use of IEEEtran.cls
%% (requires IEEEtran.cls version 1.8a or later) with an IEEE
%% Computer Society journal paper.
%%
%% Support sites:
%% http://www.michaelshell.org/tex/ieeetran/
%% http://www.ctan.org/tex-archive/macros/latex/contrib/IEEEtran/
%% and
%% http://www.ieee.org/

%%*************************************************************************
%% Legal Notice:
%% This code is offered as-is without any warranty either expressed or
%% implied; without even the implied warranty of MERCHANTABILITY or
%% FITNESS FOR A PARTICULAR PURPOSE! 
%% User assumes all risk.
%% In no event shall IEEE or any contributor to this code be liable for
%% any damages or losses, including, but not limited to, incidental,
%% consequential, or any other damages, resulting from the use or misuse
%% of any information contained here.
%%
%% All comments are the opinions of their respective authors and are not
%% necessarily endorsed by the IEEE.
%%
%% This work is distributed under the LaTeX Project Public License (LPPL)
%% ( http://www.latex-project.org/ ) version 1.3, and may be freely used,
%% distributed and modified. A copy of the LPPL, version 1.3, is included
%% in the base LaTeX documentation of all distributions of LaTeX released
%% 2003/12/01 or later.
%% Retain all contribution notices and credits.
%% ** Modified files should be clearly indicated as such, including  **
%% ** renaming them and changing author support contact information. **
%%
%% File list of work: IEEEtran.cls, IEEEtran_HOWTO.pdf, bare_adv.tex,
%%                    bare_conf.tex, bare_jrnl.tex, bare_conf_compsoc.tex,
%%                    bare_jrnl_compsoc.tex, bare_jrnl_transmag.tex
%%*************************************************************************


% *** Authors should verify (and, if needed, correct) their LaTeX system  ***
% *** with the testflow diagnostic prior to trusting their LaTeX platform ***
% *** with production work. IEEE's font choices and paper sizes can       ***
% *** trigger bugs that do not appear when using other class files.       ***                          ***
% The testflow support page is at:
% http://www.michaelshell.org/tex/testflow/


\documentclass[10pt,conference,onecolumn,compsoc]{IEEEtran}


\usepackage{hyperref}
\usepackage{enumitem}
\setlist[itemize]{leftmargin=3 cm}
\setlist[enumerate]{leftmargin=3cm}



% *** CITATION PACKAGES ***
%
\ifCLASSOPTIONcompsoc
  % IEEE Computer Society needs nocompress option
  % requires cite.sty v4.0 or later (November 2003)
  \usepackage[nocompress]{cite}
\else
  % normal IEEE
  \usepackage{cite}
\fi
% cite.sty was written by Donald Arseneau
% V1.6 and later of IEEEtran pre-defines the format of the cite.sty package
% \cite{} output to follow that of IEEE. Loading the cite package will
% result in citation numbers being automatically sorted and properly
% "compressed/ranged". e.g., [1], [9], [2], [7], [5], [6] without using
% cite.sty will become [1], [2], [5]--[7], [9] using cite.sty. cite.sty's
% \cite will automatically add leading space, if needed. Use cite.sty's
% noadjust option (cite.sty V3.8 and later) if you want to turn this off
% such as if a citation ever needs to be enclosed in parenthesis.
% cite.sty is already installed on most LaTeX systems. Be sure and use
% version 5.0 (2009-03-20) and later if using hyperref.sty.
% The latest version can be obtained at:
% http://www.ctan.org/tex-archive/macros/latex/contrib/cite/
% The documentation is contained in the cite.sty file itself.



% *** GRAPHICS RELATED PACKAGES ***
%
\ifCLASSINFOpdf
   \usepackage[pdftex]{graphicx}
 
\else
 
\fi
% graphicx was written by David Carlisle and Sebastian Rahtz. It is
% required if you want graphics, photos, etc. graphicx.sty is already
% installed on most LaTeX systems. The latest version and documentation
% can be obtained at: 
% http://www.ctan.org/tex-archive/macros/latex/required/graphics/
% Another good source of documentation is "Using Imported Graphics in
% LaTeX2e" by Keith Reckdahl which can be found at:
% http://www.ctan.org/tex-archive/info/epslatex/
%
% latex, and pdflatex in dvi mode, support graphics in encapsulated
% postscript (.eps) format. pdflatex in pdf mode supports graphics
% in .pdf, .jpeg, .png and .mps (metapost) formats. Users should ensure
% that all non-photo figures use a vector format (.eps, .pdf, .mps) and
% not a bitmapped formats (.jpeg, .png). IEEE frowns on bitmapped formats
% which can result in "jaggedy"/blurry rendering of lines and letters as
% well as large increases in file sizes.
%
% You can find documentation about the pdfTeX application at:
% http://www.tug.org/applications/pdftex









% *** PDF, URL AND HYPERLINK PACKAGES ***
%
\usepackage{url}
% url.sty was written by Donald Arseneau. It provides better support for
% handling and breaking URLs. url.sty is already installed on most LaTeX
% systems. The latest version and documentation can be obtained at:
% http://www.ctan.org/tex-archive/macros/latex/contrib/url/
% Basically, \url{my_url_here}.




\begin{document}

\title{Flash Card \\}
%
%

% received ..."  text while in non-compsoc journals this is reversed. Sigh.

\author{James Blankenship and Vrushank Mali\\% <-this % stops a space
}

\IEEEtitleabstractindextext{%
\begin{abstract}
The project is about a short quiz game. It will have levels of quiz. It will work as a multiple choice questions. The target audience is people that want to learn more about a topic or take a quiz.
\end{abstract}

}


% make the title area
\maketitle



\IEEEdisplaynontitleabstractindextext

\IEEEpeerreviewmaketitle



\section{Introduction}

The project will be a simple quiz. The group will be trying to make an application that will work as a quiz that will shuffle through questions. The project will  have a decent amount of questions and when you choose an answer it gives you information about  the topic of the question.The target for the project is people wanting to take quiz or find out information about a topic. The target audience will get information about topics when using the program. 


\subsection{Background}
There are not currently any terms that need to be explained. The reason for this idea is we wanted something information based and after running through some ideas decided on this on. The game will be a general knowledge game.
The game will have a timer as you go through the questions. It will have different difficulty levels.It will have a difficulty pattern similar to multiple choice. 

\subsection{Impacts}
There are not a lot of impacts for the project besides the grade for the project. 
If there was a impact it would be the learning that would come out of it.The impacts could be that people learn new information. 



\subsection{Challenges}
Some of the challenges for the project will likely be getting all the pieces 
working together plus something like a time function.
The easier part of the project will likely be getting the overall design of what 
we want down.

\section{Scope}
The scope of this project will be having a program that has multiple questions 
that will tell you if you got the question right. It will also tell you information about the topic in question.The stretch goals would likely be adding links, but we do not fully know how that works.The other stretch goal will be adding more questions.




\subsection{Requirements}
As part of fleshing out the scope of your requirements, you'll also need to keep in mind both your functional and non-functional requirements.  These should be listed, and explained in detail as necessary.  Use this area to explain how you gathered these requirements.

\subsubsection{Functional}
\begin{itemize}
\User needs to choose difficulty.
\User needs to have the ablity to answer  questions, and get information on topic.
\User needs to be able to make their own quiz.
\user needs to be able to store their quiz in a data base.
\User needs to be able to keep track of how many questions they got right.
\User needs to be able to edit the quiz they have added, or remove it.
\item User needs to have a private shopping cart -- this cannot be shared between users, and needs to maintain state across subsequent visits to the site
\item Users need to have website accounts -- this will help track recent purchases, keep shopping cart records, etc.
\item You'll need more than 2 of these...
\end{itemize}

\subsubsection{Non-Functional}
\begin{itemize}
\Users quiz must be stored in the data base.
\Users should be able to see  the interface and interact with it.
\item Security -- user credentials must be encrypted on disk, users should be able to reset their passwords if forgotten
\item you'll typically have fewer non-functional than functional requirements
\end{itemize}

\subsection{Use Cases}
Here are some brief examples of how the game should work.




\begin{table}
\centering
\begin{tabular}{|c|c|c|c|c|}
\hline
Use Case ID & Use Case Name & Primary Actor & Complexity & Priority \\
\hline \hline
1 & Add item to cart & Shopper & Med & 1\\
\hline
2 & Checkout & Shopper & Med & 1\\
\hline

\end{tabular}
\caption{Sample use case table}
\label{tab:useCaseIndex}
\end{table}


\begin{itemize}
\item[Use Case Number:] 1
\item[Use Case Name:] Player starts a new game
\item[Description:] A player starts the game in easy difficulty. Player will click on "Easy" button. This will load the easy level game page with first question and options for answer on the screen.
\end{itemize}


\begin{enumerate}
\item Player loads the game which will load the start menu on the screen.
\item Player will left-click on the "Easy" button.
\item The easy difficulty game will load on the screen with first question and the options for answer on the screen with a timer.
\item Player will choose one of the options for the answer by left-clicking on the button.
\item If the selected answer is correct, it will turn the answer button "green" and will show a dialogue box with more information about the topic which was in the question.
\item[Termination Outcome:] If the timer ends for first question, it will skip the first question and will show second question.
\end{enumerate}


\begin{itemize}
\item[Use Case Number:] 2
\item[Use Case Name:] Pause
\item[Description:] If the player wants to pause the game, player will click on the pause menu and it will pause the timer and will show the pause menu.
\end{itemize}


\begin{enumerate}
\item Player will left-click on the "Pause" button, which will open the pause page on screen.
\item Clicking pause button will pause the timer and will show the pause menu.
\item The pause menu shows "Resume", "Restart", "Volume", and "Exit" options for the player to choose.
\end{enumerate}


\begin{itemize}
\item[Use Case Number:] 3
\item[Use Case Name:] Restart
\item[Description:] If the player wants to restart the game, player will click on the restart button and it will restart the whole game with same difficulty and reset score and time.
\end{itemize}


\begin{enumerate}
\item After game getting paused by the player, the player left-clicks on the "Restart" button.
\item This will restart the whole game with the same difficulty level, reset score and time.
\item The player continues to play the game again.
\end{enumerate}

\begin{figure}[ht!]
\includegraphics[height=250px, width=350px]{cat1.jpg}
\caption{First picture, this is a kitten, not a use case diagram}
\label{cat1}
\end{figure}

\subsection{Interface Mockups}
At first, this will largely be completely made up, as you get further along in your project, and closer to a final product, this will typically become simple screenshots of your running application.

In this subsection, you will be showing what the screen should look like as the user moves through various use cases (make sure to tie the interface mockups back to the specific use cases they illustrate).



\section{Project Timeline}
Go back to your notes and look up a typical project development life cycle for the Waterfall approach.  How will you follow this life cycle over the remainder of this semester?  This will usually involve a chart showing your proposed timeline, with specific milestones plotted out.  Make sure you have deliverable dates from the course schedule listed, with a plan to meet them (NOTE: these are generally optimistic deadlines).

\section{Project Structure}
At first, this will be a little empty (it will need to be filled in by the time you turn in your final report).  This is your chance to discuss all of your design decisions (consider this the README's big brother).

\subsection{UML Outline}
Show the full structure of your program.  Make sure to keep on updating this section as your project evolves (you often start out with one plan, but end up modifying things as you move along).  As a note, while Dia fails miserably at generating pdfs (probably my fault), I have had much success with png files.  Make sure to wrap your images in a \texttt{figure} environment, and to reference with the \texttt{ref} command.  For example, see Figure \ref{cat2}.

\begin{figure}[ht!]
\includegraphics[scale=1.5]{cat2.jpg}
\caption{Your figures should be in the \emph{figure} environment, and have captions.  Should also be of diagrams pertaining to your project, not random internet kittens}
\label{cat2}
\end{figure}


\subsection{Design Patterns Used}
Make sure to actually use at least 2 design patterns from this class.  This is not normally part of such documentation, but largely just specific to this class -- I want to see you use the patterns!


\section{Results}
This section will start out a little vague, but it should grow as your project evolves.  With each deliverable you hand in, give me a final summary of where your project stands.  By the end, this should be a reflective section discussing how many of your original goals you managed to attain/how many desired use cases you implemented/how many extra features you added.

\subsection{Future Work}
Where are you going next with your project?
For early deliverables, what are your next steps?  (HINT: you will typically want to look back at your timeline and evaluate: did you meet your expected goals?  Are you ahead of schedule?  Did you decide to shift gears and implement a new feature?)
By the end, what do you plan on doing with this project?  Will you try to sell it?  Set it on fire?  Link to it on your resume and forget it exists?




\begin{thebibliography}{1}

\bibitem{IEEEhowto:kopka}
H.~Kopka and P.~W. Daly, \emph{A Guide to \LaTeX}, 3rd~ed.\hskip 1em plus
  0.5em minus 0.4em\relax Harlow, England: Addison-Wesley, 1999.

\end{thebibliography}



\begin{IEEEbiography}{Michael Shell}
Biography text here.
\end{IEEEbiography}

% if you will not have a photo at all:
\begin{IEEEbiographynophoto}{John Doe}
Biography text here.
\end{IEEEbiographynophoto}

% insert where needed to balance the two columns on the last page with
% biographies
%\newpage

\begin{IEEEbiographynophoto}{Jane Doe}
Biography text here.
\end{IEEEbiographynophoto}





% that's all folks
\end{document}


